\cleardoublepage

\renewcommand{\thesection}{\thechapter.\arabic{section}}
\renewcommand{\thechapter}{C}

\chapter{KI in der Wissenschaft – Werkzeug statt Wahrheit}
\phantomsection

%\addcontentsline{toc}{chapter}{Anhang C \quad KI in der Wissenschaft – Werkzeug statt\protect\\\hspace*{3.8em} Wahrheit}

%\addcontentsline{toc}{chapter}{Anhang C \quad KI in der Wissenschaft – Werkzeug statt Wahrheit}
\markboth{Anhang C}{Anhang C}
\label{anhangC:ki}

%\vspace{1em}
%\begin{center}
	%\LARGE\textbf{KI in der Wissenschaft – Werkzeug statt Wahrheit}
%\end{center}

\subsection*{Motivation}
\phantomsection
Dieses Buch entstand aus dem Wunsch, komplexe physikalische Zusammenhänge – insbesondere das Photon und seine Rolle in der modernen Physik – verständlich und fundiert darzustellen. Dabei wurde ein neues Werkzeug eingesetzt, das heute immer mehr Einzug in wissenschaftliches Arbeiten hält: \textbf{künstliche Intelligenz}\index{Künstliche Intelligenz}, konkret das Sprachmodell \textbf{ChatGPT}\index{ChatGPT} von OpenAI.

Doch wie lässt sich KI sinnvoll in der Wissenschaft\index{Wissenschaft} nutzen, ohne dass dabei Verständnis, Präzision oder Verantwortung\index{Verantwortung} verloren gehen? Und wie kann man das offenlegen, ohne die eigene wissenschaftliche Arbeit zu relativieren? Dieser Anhang gibt einen transparenten Einblick in den Entstehungsprozess dieses Buches und plädiert für einen verantwortungsvollen Umgang mit KI als Werkzeug – nicht als Wahrheit.

\subsection*{Was eine KI kann – und was nicht}
\phantomsection
KI-gestützte Sprachmodelle wie ChatGPT sind leistungsfähige Hilfsmittel beim Schreiben und Strukturieren. Sie können:
\begin{itemize}
	\item beim Formulieren erster Entwürfe helfen,
	\item komplexe Sachverhalte sprachlich glätten,
	\item Denkanstöße liefern oder Gliederungen vorschlagen,
	\item stilistische Alternativen aufzeigen.
\end{itemize}

Was sie jedoch \textbf{nicht} können:
\begin{itemize}
	\item \textbf{Verstehen}\index{Verstehen} im wissenschaftlichen Sinn,
	\item \textbf{prüfen}, ob eine Formel korrekt hergeleitet ist,
	\item \textbf{physikalische Konzepte durchdringen},
	\item \textbf{Quellen kritisch einordnen oder bewerten}.
\end{itemize}

Daher gilt: Eine KI kann \emph{unterstützen}, aber sie \textbf{kann und darf den wissenschaftlichen Erkenntnisprozess nicht ersetzen}. Wer mit KI arbeitet, muss dennoch selbst denken – und das Ergebnis stets kritisch prüfen.

\subsection*{Wie dieses Buch entstanden ist}
\phantomsection
Die Inhalte dieses Buches – von der Struktur über die physikalischen Erklärungen bis zu den mathematischen Herleitungen – wurden vom Autor konzipiert, recherchiert und verantwortet. ChatGPT kam in folgenden Bereichen unterstützend zum Einsatz:

\begin{itemize}
	\item beim \textbf{Formulieren einzelner Passagen}, z.\,B. bei Einleitungen, Zusammenfassungen oder didaktischen Abschnitten,
	\item zur \textbf{Stilüberprüfung} technischer Abschnitte,
	\item zur \textbf{Gliederungsentwicklung} in frühen Arbeitsphasen,
	\item zur Reflexion über \textbf{Verständlichkeit}\index{Verständlichkeit} und Leserführung.
\end{itemize}

Entscheidend ist: \textbf{Alle inhaltlichen Aussagen, Formeln und Interpretationen wurden vom Autor geprüft, hinterfragt, überarbeitet oder verworfen.} Keine KI war an der inhaltlichen Entwicklung der physikalischen Argumentation beteiligt.

\subsection*{Ethische Fragen und wissenschaftliche Verantwortung}
\phantomsection
Die Nutzung von KI in der Wissenschaft wirft berechtigte Fragen auf:

\begin{itemize}
	\item Wie viel darf automatisiert entstehen, ohne dass Autorschaft\index{Autorschaft} verwässert?
	\item Wie geht man mit potenziellen Fehlern\index{Fehler} um?
	\item Wie transparent muss die Nutzung offengelegt werden?
\end{itemize}

Die Antwort liegt in einem Grundprinzip wissenschaftlicher Arbeit: \textbf{Verantwortung}. Wer KI einsetzt, bleibt verantwortlich für das Ergebnis – unabhängig davon, ob einzelne Formulierungen von einem Modell vorgeschlagen wurden.

In diesem Sinne ist KI keine Autorin, sondern ein Werkzeug. Sie kann Prozesse beschleunigen, aber nicht ersetzen, was Wissenschaft im Kern ausmacht: \textbf{kritisches Denken, sorgfältiges Prüfen, methodisches Arbeiten}.

\subsection*{Empfehlungen für den Einsatz in der Forschung}
\phantomsection
Für Wissenschaftler:innen, Lehrende und Studierende ergibt sich daraus ein konstruktiver Weg:

\begin{itemize}
	\item Nutze KI \textbf{bewusst und gezielt} – für sprachliche Unterstützung, nicht für Argumentation oder Beweisführung.
	\item \textbf{Prüfe jede Aussage selbst} – gerade bei komplexen Sachverhalten.
	\item \textbf{Deklariere die Nutzung offen}, wenn es relevant ist – z.\,B. in Vorworten, Anhängen oder Einreichungserklärungen.
	\item Nutze KI nicht zur \textbf{Täuschung}\index{Täuschung} oder zum Feigenblatt, sondern als Hilfe zur besseren Darstellung deiner eigenen Gedanken.
\end{itemize}

\subsection*{Fazit: KI als Werkzeug – aber der Mensch bleibt denkend verantwortlich}
\phantomsection
Künstliche Intelligenz ist weder Ersatz noch Gegner menschlicher Erkenntnis. Sie ist ein \textbf{Werkzeug}\index{Werkzeug}, das bei der wissenschaftlichen Kommunikation helfen kann – \textbf{wenn es bewusst, reflektiert und verantwortungsvoll eingesetzt wird}.

Dieses Buch versteht sich auch in dieser Hinsicht als Beitrag zu einem neuen, aufgeklärten Umgang mit Technologie in der Wissenschaft. Nicht, weil die Technik alles kann – sondern weil wir gelernt haben, sie sinnvoll zu nutzen.

\vspace{1em}
\begin{tcolorbox}[didaktikbox, title=Leitgedanke]
	\label{box:leitgedanke}
	\small
	\textbf{KI ist nur so stark, wie der Mensch, der sie benutzt.}\\
	Sie kann Texte strukturieren, formulieren, variieren – 
	doch ohne kritisches Denken, Fachwissen und Verantwortung 
	des Autors bleibt sie ein Werkzeug ohne Sinn.
\end{tcolorbox}
