\cleardoublepage
%\appendix
\renewcommand{\thechapter}{B}
\renewcommand{\thesection}{\Alph{chapter}.\arabic{section}}
\chapter{Box Directory}
\label{anhangB}
\label{chap:boxenverzeichnis}
\thispagestyle{empty}


\section{Introduction}
\vspace{1em}
\begin{tcolorbox}[title=Physics Boxes, physikbox]
	\begin{itemize}
		\item \emph{Albert Einstein (1909)}\dotfill \pageref{box:einstein1909}
		\item \emph{Niels Bohr (1933)}\dotfill  \pageref{box:bohr1933}
		\item \emph{Richard P. Feynman (1965)}\dotfill \pageref{box:feynman1965}
		\item \emph{The Single Photon – When It Clicks}\dotfill\pageref{box:einzelphoton}
		\item \emph{How Entangled Photons Are Produced}\dotfill \pageref{box:spdc}
	\end{itemize}
\end{tcolorbox}

\vspace{1em}
\begin{tcolorbox}[title=Didactics Boxes, didaktikbox]
	\begin{itemize}
		\item \emph{Quantum Object Instead of Light Ball} \dotfill\pageref{box:lichtkugel}
		\item \emph{What Is Reality in Quantum Physics?} \dotfill\pageref{box:realitaet}
		\item \emph{What Does Entanglement Mean?} \dotfill\pageref{box:verschr}
		\item \emph{What the Experiments Teach Us} \dotfill\pageref{box:experimente}
		\item \emph{What We Take Away from Chapter I} \dotfill\pageref{box:kapitel1faz}
	\end{itemize}
\end{tcolorbox}

\section{The Path to the Light Quantum}
\vspace{1em}
\begin{tcolorbox}[title=Physics Boxes, physikbox]
	\begin{itemize}
		\item \emph{Isaac Newton (1704), Particle Theory} \dotfill\pageref{box:newton}
		\item \emph{Huygens (1690), On the Propagation of Light}\dotfill\pageref{box:huygens}
		\item \emph{Maxwell (1873), On Light and \newline Electromagnetic Waves} \dotfill\pageref{box:maxwell}
		\item \emph{What Is a Black Body?} \dotfill\pageref{box:schwarzerkoerper}
		\item \emph{Einstein (1905), Light Quantum}\dotfill\pageref{box:einstein-lichtquant}
		\item \emph{Robert A. Millikan on Einstein (1916)} \dotfill\pageref{box:millikan-einstein}
		\item \emph{Max Planck – Scientific Autobiography}\dotfill\pageref{box:planck-zitat}
	\end{itemize}
\end{tcolorbox}

\vspace{1em}
\begin{tcolorbox}[title=Didactics Boxes, didaktikbox]
	\begin{itemize}
		\item \emph{Why Did Classical Theory Fail?} \dotfill\pageref{box:klassik-versagt}
		\item \emph{An Important Historical Insight} \dotfill\pageref{box:geschichte-planck}
	\end{itemize}
\end{tcolorbox}

\vspace{1em}
\begin{tcolorbox}[title=Math Boxes, mathebox]
	\begin{itemize}
		\item \emph{Planck’s Radiation \newline Law: A Mathematical Interpolation}\dotfill\pageref{box:planck-interpolation}
	\end{itemize}
\end{tcolorbox}

\vspace{1em}
\begin{tcolorbox}[title=Hypothesis Boxes, hypobox]
	\begin{itemize}
		\item \emph{What If There Were No Quantization?} \dotfill\pageref{box:hypo-keine-quanten}
	\end{itemize}
\end{tcolorbox}

\section{Properties of the Photon}
\vspace{1em}
\begin{tcolorbox}[title=Physical Boxes,physikbox]
	
	\begin{itemize}
		\item \emph{Max Planck (1905)} \dotfill\pageref{box:planck1948}
		\item \emph{Microwave Radiation} \dotfill\pageref{box:Mikrowellenstrahlung}
		\item \emph{Green Light} \dotfill\pageref{box:grünesLicht}
		\item \emph{X-Rays} \dotfill\pageref{box:röntgenstrahlen}
		\item \emph{Photon Momentum} \dotfill\pageref{box:Photonenimpuls}
		\item \emph{Ionizing and Non-Ionizing Radiation} \dotfill\pageref{box:ionisierende}
		\item \emph{Note on Radiation Hazards} \dotfill\pageref{box:Hinweis zur Gefärdung}
		\item \emph{Conclusion on the Electromagnetic Spectrum} \dotfill\pageref{box:Fazit zum elektro}
		\item \emph{Why Don’t We See a Difference?} \dotfill\pageref{box:Warum sieht man}
		\item \emph{Properties of Photon Spin} \dotfill\pageref{box:Eigenschaften des}
		\item \emph{Comment on the Illustration} \dotfill\pageref{box:Kommentar zur Darstellung}
		\item \emph{Didactic Key Statement} \dotfill\pageref{box:didaktischerMerksatz}
		\item \emph{Superposition and Polarization} \dotfill\pageref{box:Superposition}
		\item \emph{Superposition and Polarization} \dotfill\pageref{box:Superposition und Polarisation}
		\item \emph{What Polarization Reveals About Photons} \dotfill\pageref{box:Was uns die}
	\end{itemize}
\end{tcolorbox}

\vspace{1em}
\begin{tcolorbox}[title=Mathematical Boxes,mathebox]
	\begin{itemize}
		\item \emph{Photons as Quanta of Energy} \dotfill\pageref{box:Photon als Energiequanten}
	\end{itemize}
\end{tcolorbox}

\vspace{1em}
\begin{tcolorbox}[title=Reference Boxes,hinweisbox]
	\begin{itemize}
		\item \emph{Real Image Material} \dotfill\pageref{keybox:RealesBildmaterial}
		\item \emph{Real Image Material on the Optical Tweezer} \dotfill\pageref{box:Manipulation kleiner Partikel}
		\item \emph{Conclusion on the Electromagnetic Spectrum} \dotfill\pageref{box:Fazit zum elektro}
		\item \emph{Note on the Graphic: Why Don’t \newline We See a Difference?} \dotfill\pageref{box:Warum sieht man}
	\end{itemize}
\end{tcolorbox}

\vspace{1em}
\begin{tcolorbox}[title=Hypothetical Boxes,hypobox]
	\begin{itemize}
		\item \emph{What If the Photon Had a Mass?} \dotfill\pageref{box:was wäre wenn}
	\end{itemize}
\end{tcolorbox}
\section{Experimental Confirmation of the Photon}

\vspace{1em}
\begin{tcolorbox}[title=Physical Boxes,physikbox]
	\begin{itemize}
		\item \emph{Philipp Lenard (1902)} \dotfill\pageref{box:Philipp Lenhard}
		\item \emph{Albert Einstein (1905)} \dotfill\pageref{die Erzeuguung von Licht}
		\item \emph{Robert A. Millikan (1916)} \dotfill\pageref{box:Robert A, Millikan}
		\item \emph{Albert Einstein (1905)} \dotfill\pageref{die Erscheinung der Wärm}
		\item \emph{Robert A. Millikan (1916)} \dotfill\pageref{box:einsteins gleichung passt}
		\item \emph{What Is the Work Function \( A \)?} \dotfill\pageref{bos:was ist Austrittsarbeit}
		\item \emph{What the Photon Graphic Is Meant to Show} \dotfill\pageref{box:was die photonengrafik}
		\item \emph{What Antibunching Demonstrates} \dotfill\pageref{box:wasAntibunching}
		\item \emph{What the HOM Effect Demonstrates} \dotfill\pageref{box:HOM-Effekt}
	\end{itemize}
\end{tcolorbox}

\vspace{1em}
\begin{tcolorbox}[title=Mathematical Boxes,mathebox]
	\begin{itemize}
		\item \emph{Compton Formula} \dotfill\pageref{box:comptonFormel}
	\end{itemize}
\end{tcolorbox}

\vspace{1em}
\begin{tcolorbox}[title=Didactic Boxes,didaktikbox]
	\begin{itemize}
		\item \emph{Didactic Clarification} \dotfill\pageref{box:didaktischeKlarstellung}
		\item \emph{Both Wave and Particle Properties} \dotfill\pageref{box:wellen}
		\item \emph{Conclusion: An Apparently Paradoxical Behavior} \dotfill\pageref{box:Fazit ein scheinbarer}
	\end{itemize}
\end{tcolorbox}

\vspace{1em}
\begin{tcolorbox}[title=Reference Boxes,hinweisbox]
	\begin{itemize}
		\item \emph{Conclusion} \dotfill\pageref{box:fazit der photo}
		\item \emph{What This Illustration Shows} \dotfill\pageref{box:was diese Darstellun}
		\item \emph{What the Experiments Reveal About Light} \dotfill\pageref{box:was die Experimente}
	\end{itemize}
\end{tcolorbox}

\vspace{1em}
\begin{tcolorbox}[title=Hypothetical Boxes,hypobox]
	\begin{itemize}
		\item \emph{Key Idea} \dotfill\pageref{box:schlüsselidee}
	\end{itemize}
\end{tcolorbox}
\section{The Photon in Quantum Electrodynamics (QED)}

\vspace{1em}
\begin{tcolorbox}[title=Physical Boxes, physikbox]
	\begin{itemize}
		\item \textbf{What Is Quantum Electrodynamics?} \dotfill \pageref{box:was ist quantenelektro}
		\item \textbf{What Is the Field Formalism?} \dotfill \pageref{box:was ist Feldformalismus}
		\item \textbf{What Does Gauge Symmetry Mean?} \dotfill \pageref{box:was bedeutet Eichsy}
		\item \textbf{Consequences of Gauge Symmetry} \dotfill \pageref{box:folgen der Eichsy}
		\item \textbf{Virtual Particles in the Quantum Vacuum} \dotfill \pageref{box:virtuelle-teilchen}
		\item \textbf{Virtual Photons as Force Mediators} \dotfill \pageref{box:Virtuelle Photonen als kraftvermittler}
		\item \textbf{What a Feynman Diagram Really Shows} \dotfill \pageref{box:Was ein Feynman-Diagramm}
		\item \textbf{Virtual Photon} \dotfill \pageref{box:virtuelles Photon}
		\item \textbf{Real Photons} \dotfill \pageref{box:Reale Photonen}
		\item \textbf{No QED Without Virtual Photons} \dotfill \pageref{box:Ohne virtuelle Photonen keine}
		\item \textbf{Field Theory Instead of Particle Mechanics} \dotfill \pageref{box:Feldtheorie statt Teilchenmechanik}
		\item \textbf{Photon Field from the Lagrangian Density} \dotfill \pageref{box:Photonenfeld aus der Lagrangedichte}
		\item \textbf{What Is a Photon in QED?} \dotfill \pageref{box:Warum ist ein Photon in der QED}
		\item \textbf{Physical Significance} \dotfill \pageref{box:physikalische Bedeutung}
	\end{itemize}
\end{tcolorbox}

\vspace{1em}
\begin{tcolorbox}[title=Mathematical Boxes, mathebox]
	\begin{itemize}
		\item \textbf{Structure of the QED Lagrangian Density} \dotfill \pageref{box:Sufbau der QED-Langrangedichte}
		\item \textbf{Coupling from Principle} \dotfill \pageref{box:Kopplung aus Prinzip}
		\item \textbf{Feynman Rules of QED (Simplified)} \dotfill \pageref{box:Feynman-Regeln der QED}
	\end{itemize}
\end{tcolorbox}

\vspace{1em}
\begin{tcolorbox}[title=Didactic Boxes, didaktikbox]
	\begin{itemize}
		\item \textbf{Does Virtual Mean Less Real?} \dotfill \pageref{box:virtuell-denkfehler}
		\item \textbf{What Does the Feynman Diagram\newline Really Show?} \dotfill \pageref{box:Was zeigt das Feynman-Diagramm wirklich}
		\item \textbf{No More “Invisible Force” Needed} \dotfill \pageref{box:unsichtbare Kraft}
		\item \textbf{Diagram Does Not Equal Reality} \dotfill \pageref{boxx:Diagramm ist nicht gleich realität}
		\item \textbf{Real or Virtual Photons in the Diagram} \dotfill \pageref{box:Reale oder virtuelle Photonen}
		\item \textbf{Quantum Electrodynamics\newline as a Success Model} \dotfill \pageref{box:Die Quantenelekrodynamik}
		\item \textbf{No Particle Trajectories in the Diagram} \dotfill \pageref{box: Keine Teilchenbahn im Diagramm}
		\item \textbf{Why Not Simply Classical?} \dotfill \pageref{box:Warum nicht einfach klassisch?}
		\item \textbf{From Lagrangian Term to Feynman Vertex} \dotfill \pageref{box:Vom Lagrange-Term zum Feynmann-Vertax}
		\item \textbf{The Force Arises from the Derivative} \dotfill \pageref{box:Die Kraft entsteht aus dem Ableiter}
		\item \textbf{Why the Rules Work} \dotfill \pageref{box:Warum die Regeln funktionieren}
		\item \textbf{Why Doesn’t the Photon Have a\newline Spin-3 State?} \dotfill \pageref{box:Warum hat das Photon keinen  Spin-3-Zustand}
		\item \textbf{Why This Matters} \dotfill \pageref{box:Warum ist wichtig}
	\end{itemize}
\end{tcolorbox}

\vspace{1em}
\begin{tcolorbox}[title=Reference Boxes, hinweisbox]
	\begin{itemize}
		\item \textbf{On-Shell Condition} \dotfill \pageref{box:On-shell-Bedingung}
		\item \textbf{Indirect Evidence for Virtual Photons} \dotfill \pageref{box:Nachweis virtueller Photonen}
		\item \textbf{Loop Diagrams and Precision Effects} \dotfill \pageref{box:Schleifendiagramme}
		\item \textbf{Note for Readers} \dotfill \pageref{box:Hinweis füe Leser}
		\item \textbf{Outlook on Chapter VI} \dotfill \pageref{box:Ausblick auf Kapitel 6}
	\end{itemize}
\end{tcolorbox}

\vspace{1em}
\begin{tcolorbox}[title=Hypothetical Boxes, hypobox]
	\begin{itemize}
		\item \textbf{What If Light Were Not Isotropic?} \dotfill \pageref{box:was wäre nicht isotop}
	\end{itemize}
\end{tcolorbox}

\vspace{1em}
\begin{tcolorbox}[title=Warning Boxes, warnbox]
	\begin{itemize}
		\item \textbf{Do Not Take Feynman Diagrams Literally!} \dotfill \pageref{box:Warnung}
	\end{itemize}
\end{tcolorbox}


\section{Applications of the Photon}
\vspace{1em}

\begin{tcolorbox}[title=Physical Boxes, physikbox]
	\begin{itemize}
		\item \emph{Stimulated Emission as the Basis of the Laser} \dotfill\pageref{box:grundlagedeslaser}
		\item \emph{Physical Terms} \dotfill\pageref{box:begriffe}
		\item \emph{Photons in PET} \dotfill\pageref{box:PET}
		\item \emph{Total Internal Reflection in Optical Fibers} \dotfill\pageref{box:glasfaser}
		\item \emph{Why Don’t Some Photons Reach Earth?} \dotfill\pageref{box:photonen auf erde}
		\item \emph{Spectral Lines and Redshift} \dotfill\pageref{box:spektrallinien}
		\item \emph{Photons as Tools for Measuring Spacetime Curvature} \dotfill\pageref{box:messwerkzeug}
	\end{itemize}
\end{tcolorbox}

\vspace{1em}
\begin{tcolorbox}[title=Didactic Boxes, didaktikbox]
	\begin{itemize}
		\item \emph{Diversity of Laser Types – An Overview} \dotfill\pageref{box:Typenvielfalt von Lasern}
		\item \emph{Applications of Lasers – Technology (Overview)} \dotfill\pageref{box:lasertechnik}
		\item \emph{Applications of Lasers – Research (Overview)} \dotfill\pageref{box:laser-app-forschung}
		\item \emph{Definition: Avalanche} \dotfill\pageref{box:avalanche}
		\item \emph{Definition: Scintillator} \dotfill\pageref{box:szintillator}
		\item \emph{What Counts as a Photon?} \dotfill\pageref{box:photonenzaehlung}
		\item \emph{Definition: Annihilation} \dotfill\pageref{box:annihilation}
		\item \emph{Definition: Radiopharmaceutical} \dotfill\pageref{box:radiopharmakon}
		\item \emph{Advantage of Optical Methods} \dotfill\pageref{box:optisches Verfahren}
		\item \emph{Didactic Explanation: Interference in OCT} \dotfill\pageref{box:interferenz_oct}
		\item \emph{What Makes QKD Secure?} \dotfill\pageref{box:qkd}
		\item \emph{How Does QKD Work (e.g., BB84)?} \dotfill\pageref{box:wie funktioniert QKD}
		\item \emph{What Does a Spectrum Show?} \dotfill\pageref{box:was zeigt spektrum}
		\item \emph{How Does an Interferometer Work?} \dotfill\pageref{box:interferometer}
		\item \emph{Why Are Photons Measurable with Such Precision?} \dotfill\pageref{box:photonen_genau}
	\end{itemize}
\end{tcolorbox}

\vspace{1em}
\begin{tcolorbox}[title=Reference Boxes, hinweisbox]
	\begin{itemize}
		\item \emph{Didactic Comparison: SPAD vs. PMT} \dotfill\pageref{box:vergleich SPAD}
		\item \emph{Note on the Importance of Detection Technology} \dotfill\pageref{box:detektionstechnologie}
		\item \emph{Laser Parameters} \dotfill\pageref{box:laserparameter}
		\item \emph{Future of Photonic Communication} \dotfill\pageref{box:Zukunft Kommunikation}
	\end{itemize}
\end{tcolorbox}
\section{Photons and the Future of Physics}
\vspace{1em}
\begin{tcolorbox}[title=Physical Boxes, physikbox]
	\begin{itemize}
		\item \emph{What Makes a Photon an Information Carrier?} \dotfill\pageref{box:photon_information}
		\item \emph{The Hong–Ou–Mandel Dip Phenomenon} \dotfill\pageref{box:hong_ou_mandel}
		\item \emph{Core Principle of Quantum Cryptography} \dotfill\pageref{box:qcrypto_prinzip}
		\item \emph{Photonic Circuits vs. Electronic Circuits} \dotfill\pageref{box:photon_vs_electron}
		\item \emph{Why Photons Are Interesting for Logic Circuits} \dotfill\pageref{box:optlogik_vorteile}
		\item \emph{Photons as Messengers of the Laws of Nature} \dotfill\pageref{box:photonen_grundlagen}
	\end{itemize}
\end{tcolorbox}

\vspace{1em}
\begin{tcolorbox}[title=Didactic Boxes, didaktikbox]
	\begin{itemize}
		\item \emph{From Electronics to Photonics} \dotfill\pageref{box:optlogik_didaktik}
		\item \emph{Photonic AND Gate in a Mach–Zehnder\newline Interferometer} \dotfill\pageref{box:mzi_and}
	\end{itemize}
\end{tcolorbox}

\vspace{1em}
\begin{tcolorbox}[title=Hypothetical Boxes, hypobox]
	\begin{itemize}
		\item \emph{What If Optical Computers Replaced Electronics?} \dotfill\pageref{box:optlogik_zukunft}
		\item \emph{What If the Photon Were Not the \newline Only Massless Boson?} \dotfill\pageref{box:photon_neue_physik}
		\item \emph{What If We Could Fully Control Photons?} \dotfill\pageref{box:hypo_kapVII}
	\end{itemize}
\end{tcolorbox}

\vspace{1em}
\begin{tcolorbox}[title=Reference Boxes, hinweisbox]
	\begin{itemize}
		\item \emph{What Does “Indistinguishability” Mean?} \dotfill\pageref{box:indistinguishability}
		\item \emph{What Does “Photonics” Mean?} \dotfill\pageref{box:photonics_definition}
	\end{itemize}
\end{tcolorbox}


\section{The Photon in the Standard Model of Particle Physics}
\vspace{1em}

\begin{tcolorbox}[title=Physical Boxes, physikbox]
	\begin{itemize}
		\item \emph{Massless Photon and Spin Effects\newline over Long Distances} \dotfill\pageref{box:photon_spin_reichweite}
	\end{itemize}
\end{tcolorbox}

\vspace{1em}
\begin{tcolorbox}[title=Didactic Boxes, didaktikbox]
	\begin{itemize}
		\item \emph{U(1) Explained Intuitively} \dotfill\pageref{box:u1_kreis}
		\item \emph{From \(W^3\) and \(B\) to the Photon} \dotfill\pageref{box:weinberg_mischung}
		\item \emph{Didactic Conclusion: The Photon\newline in the Standard Model} \dotfill\pageref{box:didaktik_kapVIII}
	\end{itemize}
\end{tcolorbox}

\vspace{1em}
\begin{tcolorbox}[title=Reference Boxes, hinweisbox]
	\begin{itemize}
		\item \emph{Masslessness and Range} \dotfill\pageref{box:reichweite_masselos}
		\item \emph{Comparison of Gauge Bosons} \dotfill\pageref{box:eichbosonen_vergleich}
	\end{itemize}
\end{tcolorbox}

\vspace{1em}
\begin{tcolorbox}[title=Hypothetical Boxes, hypobox]
	\begin{itemize}
		\item \emph{What If the Standard Model Were\newline Only a Transitional Step?} \dotfill\pageref{Merksatz zum Photon}
	\end{itemize}
\end{tcolorbox}


\section{Appendix C}
\vspace{1em}
\begin{tcolorbox}[title=Didactic Boxes, didaktikbox]
		\label{box:guiding_principle}
	\begin{itemize}
		\item \emph{Guiding Principle} \dotfill	\pageref{box:guiding_principle}
	\end{itemize}
\end{tcolorbox}
